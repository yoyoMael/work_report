\documentclass[a4papper]{article}
\title{}
\usepackage{color}
\usepackage{ctex}
\usepackage{amssymb}
\usepackage{amsmath}
\usepackage{upgreek}
\usepackage{mathtools}
\usepackage{subfigure}
\usepackage{float}
\usepackage{indentfirst} %段落自动缩进
\usepackage{geometry}  %页边距
\usepackage{subfigure}
\usepackage{booktabs}
\usepackage{txfonts}
\usepackage{setspace}
\usepackage{CJK,CJKnumb}
\usepackage{fancyhdr} % 添加页眉页脚
\pagestyle{fancy}%设置页眉和页脚
\lhead{}
\chead{}
\lfoot{}
\cfoot{}
\rfoot{\thepage}
%\renewcommand{\headrulewidth}{0pt} %改为0pt即可去掉页眉下面的横线
%\renewcommand{\footrulewidth}{0pt} %改为0pt即可去掉页脚上面的横线 0.4pt
\usepackage[raggedleft]{titlesec} %设置标题
\usepackage{textcomp}

\newcommand{\kai}{\CJKfamily{kai}} 

\renewcommand{\figurename}{Fig.}
\renewcommand{\contentsname}{目录}
\renewcommand{\thefigure}{\arabic{figure}}
\renewcommand{\tablename}{Table 1.}
\renewcommand{\thesubfigure}{(\alph{subfigure}}

\titleformat{\section}{\center\Huge\bfseries}{\thesection .}{.5em}{}%定义section样式
\titleformat{\subsection}{\raggedright\Large\bfseries}{\thesubsection .}{.5em}{}
\titleformat{\subsubsection}{\raggedright\large\bfseries}{\thesubsubsection .}{.5em}{}

\bibliographystyle{plain}

\geometry{left=3cm,top=3cm,bottom=4cm,right=3cm}
\title{第二周}
\author{
Wu You\\ 
}
\begin{document}
\maketitle
\section*{摘要}

\newpage
\tableofcontents
\thispagestyle{empty}%取消此页页眉和页脚
\newpage
\setcounter{page}{1}%将此页页码计数设置为1
\setcounter{section}{0}%在此处将section计数设置为0,下个section将为1号

\section{安全状态画像功能简介}
\par\setlength{\parindent}{2em} %设置段落缩进
安全状态画像是基于子平台结合现有数据,并通过综合分析最终形成的具有一定行为特点和标签的画像集合,主要包括资产画像、威胁画像、脆弱性画像和安全事件画像,帮助用户全面了解平台现有资产安全状态、风险状态,并提供了直观清晰的信息展示方式。
\section{网络安全画像简介}
\par\setlength{\parindent}{2em} %设置段落缩进
画像技术大多应用于用户分析,可服务于商品推荐、用户行为预测、营销策略的制定等场景,而对于网络安全的状态,我们也可以使用画像技术来进行描述,以便于更好地呈现网络态势,有助于形成更好的防御策略,并实现威胁预警。网络安全画像,实际上就是利用数据对网络的各方面,包括漏洞、弱点、防御能力等,进行定性或定量评估,并将评估结果用直观的、易于理解的方式对用户进行呈现。在安全画像的构建过程中,主要使用数据挖掘技术以及数据融合技术
\begin{enumerate}
\item  
\end{enumerate}
\section{国内外研究现状}
\par\setlength{\parindent}{2em} %设置段落缩进

\subsection{决策树的构建过程}
\par\setlength{\parindent}{2em} %设置段落缩进
决策树算法通过结合训练样例的目标值,对训练样例的各个属性依次分类,每个属性为一个节点,若节点中含有目标属性值不同的样例,则进一步进行划分,最终形成一个树形结构,该树可为二叉树或多叉树。
构建树的首个问题是选择最佳的初始分类属性,也就是根节点最佳属性的选择。
\subsubsection{信息增益}
\par\setlength{\parindent}{2em} %设置段落缩进
对于概念学习,信息熵的定义如下:
\begin{equation*}
Entropy(S) = -p_{\oplus} \cdot log_2 p_{\oplus} - p_{\ominus} \cdot log_2 p_{\ominus}
\end{equation*}
\par\setlength{\parindent}{2em} %设置段落缩进
其中$p_{\oplus}$和$p_{\ominus}$分别为被该属性分为正例和反例的样本所占比例。
进一步地,对于目标属性有c个取值的概念,信息熵为:
\begin{equation*}
Entropy(S) = \sum_{i = 1}^{c} -p_{i} \cdot log_2 p_{i}
\end{equation*}
\par\setlength{\parindent}{2em} %设置段落缩进
对任意属性,其信息熵最大值为$log_2 c$最小值为0。信息熵描述了样本的均一性。有了信息熵,之后可以对信息增益进行定义,对样本集合S,某一属性A的信息增益定义如下:
\begin{equation*}
Gain(S, A) = Entropy(S) - \sum_{v \in Values(A)}\frac{\left|S_v\right|}{\left|S\right|}Entropy(S_v)
\end{equation*}
\par\setlength{\parindent}{2em} %设置段落缩进
该属性刻画了属性A对样本S的分类效果,信息增益越大,表明样例信息熵降低越多,也就意味着分类后的样本更加一致,在进行最佳分类属性的选择时,以信息增益取最大值为标准。
\subsubsection{决策树修剪}
在对样本的学习过程中,由于噪声或特殊样例的存在,通常会造成过拟合的情况发生,即该树在训练样本集合上表现良好,但在对新的实例进行分类时准确率欠佳(泛化精度低),为提高泛化精度,需要额外的处理,主要有两种方式提高泛化精度:
\par\setlength{\parindent}{2em} %设置段落缩进
\begin{itemize}\setlength{\itemindent}{2em}%使列表也缩进
\item 提前停止树的增长
\item 对树进行后修剪
\end{itemize}
\par\setlength{\parindent}{2em} %设置段落缩进
对于以上两种方法,有一个共同的问题:确定其遵循的准则,也就是何时停止或对树修剪到何种程度,对此问题有以下几种解决方案:
\begin{itemize}\setlength{\itemindent}{2em}%使列表也缩进
\item 用新的样例来评估效果
\item 对增长或修剪操作的效果进行估计
\item 用编码来衡量
\end{itemize}
\par\setlength{\parindent}{2em} %设置段落缩进
其中,第一种方法最为常用,我们可以将训练样例分为训练集和验证集,比例为1:1或2:1,但验证集要足够大。(k-折交叉验证与此类似,此处应为2-折交叉验证)下面是两种树的修剪方法:
\begin{enumerate}\setlength{\itemindent}{2em}
  \item 错误率降低修剪
  \item 规则后修剪
\end{enumerate}
\par\setlength{\parindent}{2em} %设置段落缩进
错误率降低修剪是将以某一节点为根的树全部移除,并将属于该子树的最常见目标属性值赋予该节点,但仅在可以降低错误率时进行此操作。规则后修剪方法首先令树生长到尽可能拟合训练数据,之后将树转化为等价的规则集合,为根节点到每个叶节点的路径创建规则($ \wedge $),之后对这些规则进行泛化以提高其精度,这一方法使得修剪更为灵活,并且对人来说,规则更加容易理解。
\subsubsection{一些特殊属性的处理}
\indent
\textbf{连续值属性}
\par\setlength{\parindent}{2em} %设置段落缩进
决策树在处理离散值时,可以将其取值范围划分为多个区间,从而像对待离散值属性一样处理该属性。\\
\indent\textbf{有大量取值的属性}
\par\setlength{\parindent}{2em} %设置段落缩进
例如日期这类属性,拥有大量的不同属性值,假设在某一节点以日期为1号对样例进行划分,会使得样例被划分为一大一小两部分,此时该属性的信息增益将大于其他属性,但实际上该属性分类效果很差,此时需要用其他标准来代替信息增益,信息增益比率是一个可用的标准。首先定义\emph{分裂信息}(Splitinformation):
\begin{equation*}
  splitinformation(S,A) = \sum_{i=1}^{c}\frac{|S_i|}{S}\cdot log \frac{|S_i|}{S}
\end{equation*}
\par\setlength{\parindent}{2em} %设置段落缩进
信息增益比率的定义如下:
\begin{equation*}
  Gainratio(S,A) = \frac{Gain{S,A}}{splitinformation(S,A)}
\end{equation*}
实际上,\emph{分裂信息}就是S关于A的熵,当A的取值数量较多时,其\emph{分裂信息}也较大,信息增益比率的值会相应减小,这相当于对取值较多的属性进行了惩罚。\\\\
\textbf{处理缺少的属性}
\par\setlength{\parindent}{2em} %设置段落缩进
对于缺少某属性的样例,可赋予其最常见值,或按照已观察到的该属性取值比例对这些样例进行赋值。\\\\
\textbf{处理不同代价的属性}
\section{贝叶斯方法}
\par\setlength{\parindent}{2em} %设置段落缩进
贝叶斯方法利用先验知识和观察数据一同决定假设的最终概率,可以对假设做不确定性的预测,该方法也可用于衡量其他算法的最优决策。
\subsection{贝叶斯法则}
\par\setlength{\parindent}{2em} %设置段落缩进
对某一假设h和训练样例D,有如下的贝叶斯公式:
\begin{equation*}
  P(h|D) = \frac{P(D|h) \cdot P(h)}{P(D)}
\end{equation*}
\par\setlength{\parindent}{2em} %设置段落缩进
其中,$P(h)$是h的先验概率,基于现有知识得到,$P(D|h)$是在我们认为h成立的情况下,观察到D的概率,而$P(D)$是D的先验概率,等式左边的$P(h|D)$为假设h的后验概率,即观察到D时,h成立的概率。\emph{极大后验概率假设}(MAP)定义为使后验概率最大的假设:
\begin{align*}
h_{MAP} & = \mathop{argmax}_{h \in H} P(h|D)\\
& = \mathop{argmax}_{h \in H} P(D|h) P(h)
\end{align*}
\par\setlength{\parindent}{2em}
有时候我们会给每个假设赋予相同的先验概率,这时,我们只需要根据$P(D|h)$来决定最大可能的假设,$P(D|h)$常称为h成立时样本D的似然度,使该概率最大的假设称为\emph{极大似然假设}(ML):
\begin{align*}
  h_{ML} = \mathop{argmax}_{h \in H} P(D|h)
\end{align*}
\subsection{贝叶斯理论的应用}
\par\setlength{\parindent}{2em}
对于具有一组属性${a_1, a_2, ...,a_n}$的实例来说,其\emph{极大后验假设}为:
\begin{equation*}
  h_{MAP} = \mathop{argmax}_(v_j \in V) P(v_j|a_1, a_2, a_3, ..., a_n)P(v_j)
\end{equation*}
\par\setlength{\parindent}{2em}
朴素贝叶斯分类器基于以下假设:各个属性值的取值是互相独立的。因此,其\emph{极大后验假设}可以进一步表示为:
\begin{align*}
   h_{MAP} & = v_{NB}\\
   & = \mathop{argmax}_{v_j \in V} P(v_j|a_1, a_2, a_3, ..., a_n)P(v_j) \\
   & = \mathop{argmax}_{v_j \in V} \prod _{i = 1} ^n P(a_i|v_j) P(v_j)
\end{align*}
\par\setlength{\parindent}{2em}
并且
\begin{equation*}
  P(a_i|v_j) = \frac{n_c + mp}{n+m}
\end{equation*}
\bibliography{reference}
\end{document}